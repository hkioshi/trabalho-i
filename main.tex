% abtex2-modelo-artigo.tex, v-1.9.2 laurocesar
% Copyright 2012-2014 by abnTeX2 group at http://abntex2.googlecode.com/ 
%

% ------------------------------------------------------------------------
% ------------------------------------------------------------------------
% abnTeX2: Modelo de Artigo Acadêmico em conformidade com
% ABNT NBR 6022:2003: Informação e documentação - Artigo em publicação 
% periódica científica impressa - Apresentação
% ------------------------------------------------------------------------
% ------------------------------------------------------------------------

\documentclass[
	% -- opções da classe memoir --
	article,			% indica que é um artigo acadêmico
	11pt,				% tamanho da fonte
	oneside,			% para impressão apenas no verso. Oposto a twoside
	a4paper,			% tamanho do papel. 
	% -- opções da classe abntex2 --
	%chapter=TITLE,		% títulos de capítulos convertidos em letras maiúsculas
	%section=TITLE,		% títulos de seções convertidos em letras maiúsculas
	%subsection=TITLE,	% títulos de subseções convertidos em letras maiúsculas
	%subsubsection=TITLE % títulos de subsubseções convertidos em letras maiúsculas
	% -- opções do pacote babel --
	english,			% idioma adicional para hifenização
	brazil,				% o último idioma é o principal do documento
	sumario=tradicional
	]{abntex2}


% ---
% PACOTES
% ---

% ---
% Pacotes fundamentais 
% ---
\usepackage{lmodern}			% Usa a fonte Latin Modern
\usepackage[T1]{fontenc}		% Selecao de codigos de fonte.
\usepackage[utf8]{inputenc}		% Codificacao do documento (conversão automática dos acentos)
\usepackage{indentfirst}		% Indenta o primeiro parágrafo de cada seção.
\usepackage{nomencl} 			% Lista de simbolos
\usepackage{color}				% Controle das cores
\usepackage{graphicx}			% Inclusão de gráficos
\usepackage{microtype} 			% para melhorias de justificação
\usepackage{ctable}
% ---
		
% ---
% Pacotes adicionais, usados apenas no âmbito do Modelo Canônico do abnteX2
% ---
\usepackage{lipsum}				% para geração de dummy text
% ---
		
% ---
% Pacotes de citações
% ---
\usepackage[brazilian,hyperpageref]{backref}	 % Paginas com as citações na bibl
\usepackage[alf]{abntex2cite}	% Citações padrão ABNT
% ---

% ---
% Configurações do pacote backref
% Usado sem a opção hyperpageref de backref
\renewcommand{\backrefpagesname}{Citado na(s) página(s):~}
% Texto padrão antes do número das páginas
\renewcommand{\backref}{}
% Define os textos da citação
\renewcommand*{\backrefalt}[4]{
	\ifcase #1 %
		Nenhuma citação no texto.%
	\or
		Citado na página #2.%
	\else
		Citado #1 vezes nas páginas #2.%
	\fi}%
% ---

% ---
% Informações de dados para CAPA e FOLHA DE ROSTO
% ---
\titulo{Artigo de Bels e Decibels}
\autor{Equipe  Trabalhos \and henrique }
\local{Brasil}
\data{2022, v-1.0.0}
% ---

% ---
% Configurações de aparência do PDF final

% alterando o aspecto da cor azul
\definecolor{blue}{RGB}{41,5,195}

% informações do PDF
\makeatletter
\hypersetup{
     	%pagebackref=true,
		pdftitle={\@title}, 
		pdfauthor={\@author},
    	pdfsubject={Modelo de artigo científico com abnTeX2},
	    pdfcreator={LaTeX with abnTeX2},
		pdfkeywords={abnt}{latex}{abntex}{abntex2}{atigo científico}, 
		colorlinks=true,       		% false: boxed links; true: colored links
    	linkcolor=blue,          	% color of internal links
    	citecolor=blue,        		% color of links to bibliography
    	filecolor=magenta,      		% color of file links
		urlcolor=blue,
		bookmarksdepth=4
}
\makeatother
% --- 

% ---
% compila o indice
% ---
\makeindex
% ---

% ---
% Altera as margens padrões
% ---
\setlrmarginsandblock{3cm}{3cm}{*}
\setulmarginsandblock{3cm}{3cm}{*}
\checkandfixthelayout
% ---

% --- 
% Espaçamentos entre linhas e parágrafos 
% --- 

% O tamanho do parágrafo é dado por:
\setlength{\parindent}{1.3cm}

% Controle do espaçamento entre um parágrafo e outro:
\setlength{\parskip}{0.2cm}  % tente também \onelineskip

% Espaçamento simples
\SingleSpacing

% ----
% Início do documento
% ----
\begin{document}

% Retira espaço extra obsoleto entre as frases.
\frenchspacing 

% ----------------------------------------------------------
% ELEMENTOS PRÉ-TEXTUAIS
% ----------------------------------------------------------

%---
%
% Se desejar escrever o artigo em duas colunas, descomente a linha abaixo
% e a linha com o texto ``FIM DE ARTIGO EM DUAS COLUNAS''.
% \twocolumn[    		% INICIO DE ARTIGO EM DUAS COLUNAS
%
%---
% página de titulo
\maketitle

% resumo em português
\begin{resumoumacoluna}
    Este artigo falara sobre a definição de bels e decibels, que são unidades de mediadas de som e quais são suas possiveis utilidades na acustuca e telecomunicações 
 
 \vspace{\onelineskip}
 
 \noindent
 \textbf{Palavras-chaves}: bels. decibels. som.
\end{resumoumacoluna}

% ]  				% FIM DE ARTIGO EM DUAS COLUNAS
% ---

% ----------------------------------------------------------
% ELEMENTOS TEXTUAIS
% ----------------------------------------------------------
\textual

% ----------------------------------------------------------
% Introdução
% ----------------------------------------------------------
\section*{Introdução}
\addcontentsline{toc}{section}{Introdução}

        O decibel (dB) é uma medida da razão entre duas quantidades,
    sendo usado para uma grande variedade de medições em acústica,
    física, eletrônica e telecomunicações. Por ser uma razão entre duas
    quantidades iguais o decibel é uma unidade de medida adimensional
    semelhante a percentagem. O dB usa o logaritmo decimal (log10) para
    realizar a compressão de escala. Um exemplo típico de uso do dB é na
    medição do ganho/perda de potência em um sistema. Além do uso do
    dB como medida relativa, também existem outras aplicações na
    medidas de valores absolutos tais como potência e tensão entre outros
    (dBm, dBV, dBu). O emprego da subunidade dB é para facilitar o seu
    uso diário (Um decibel (dB) corresponde a um décimo de bel (B)). 

% ----------------------------------------------------------
% Seção de explicações
% ----------------------------------------------------------
\section{origem}

    O bel foi inventado por engenheiros do Bell Labs para quantificar a
redução no nível acústico sobre um cabo telefônico padrão com 1 milha
de comprimento. Originalmente era chamado de unidade de
V.2006 1
PRINCÍPIO DE SISTEMAS DE TELECOMUNICAÇÕES IFSC
transmissão ou TU, mas foi renomeado entre 1923 e 1924 em
homenagem ao fundador do laboratório Alexander Graham Bell.


\section{Vantagens do uso do decibels}

    As vantagens do uso do decibel são:
\begin{itemize}
    \item   É mais conveniente somar os ganhos em decibéis em estágios
    sucessivos de um sistema do que multiplicar os seus ganhos
    lineares.
    
    \item     Faixas muito grandes de razões de valores podem ser expressas em
    decibéis em uma faixa mais moderada possibilitando uma melhor
    visualização dos valores grandes e pequenos.
    
    \item   Na acústica o decibel usado como uma escala logarítmica da razão
    de intensidade sonora se ajusta melhor a intensidade percebida pelo
    ouvido humano. O aumento do nível de intensidade em decibéis
    corresponde aproximadamente ao aumento percebido em qualquer
    intensidade, fato conhecido com a Lei de potências de Stevens. Por
    exemplo, um humano percebe um aumento de 90 dB para 95 dB
    como sendo o mesmo que um aumento de 20 dB para 25 dB.
\end{itemize}


\section{Usos do Bel/Decibel}
\subsection{Razões de Potência}
    O cálculo da relação de potência em dB GdB entre dois valores de
    potência corresponde ao ganho de potência, sendo dado por
    
    \begin{equation}
        {\displaystyle G_{dB}} = 10\log_{10}(\frac{{\displaystyle P_{1}}}{{\displaystyle P_{0}}})
    \end{equation}
    
         Onde P0 e P1 são níveis de potências absolutas expressas na mesma unidade (W, mW, pW, etc), e GdB é a razão entre as potências (ganho) expressa em dB. relação entre 2 potências é conhecida como ganho linear

    \begin{equation}
        {\displaystyle G_{W/W}} = \frac{{\displaystyle P_{1}}}{{\displaystyle P_{0}}}
    \end{equation}

        O recíproco do ganho é conhecido como atenuação
    
    \begin{equation}
        {\displaystyle A_{W/W}} = \frac{1}{{\displaystyle G_{W/W}}}
    \end{equation}

        Em decibéis a atenuação é dada por
        
     \begin{equation}
        {\displaystyle A_{dB}} = 10\log_{10}(\frac{{\displaystyle P_{1}}}{{\displaystyle P_{2}}})^{-1} =  {\displaystyle -G_{dB}}
    \end{equation}
    
        Como o dB é uma unidade de comparação de níveis de potência. Não é
    correto dizer que uma potência vale X dB e sim que uma potência P1
     é
    X dB maior (GdB >0) ou menor (GdB <0) que a outra potência P0
    .
    Quando P1
     representar a potência de um sinal (S - Signal) e P0
     a
    potência de um ruído (N - Noise) designamos a razão entre as
    potências de razão sinal/ruído (SNR – Signal Noise Ratio).
    
        A razão entre tensões também pode ser expressas em decibéis através
    da equação:
    
    \begin{equation}
    {\displaystyle G_{dB}} = 10\log_{10}(\frac{{\displaystyle V_{2 1}}/{\displaystyle Z_{0}}}{{\displaystyle V_{2 0}}/{\displaystyle Z_{0}}}) = 20\log_{10}|\frac{{\displaystyle V_{1}}}{{\displaystyle V_{0}}}|
        
    \end{equation}
    
        Essa relação de tensões em dB é equivalente a relação de potencias
    entre os pontos se as impedâncias Z0
     e Z1
     forem iguais. No entanto se
    forem diferentes, é incorreto utilizar essa medida. Veja porque abaixo:
    
    \begin{equation}
        {\displaystyle G_{dB}} =10\log_{10}(\frac{{\displaystyle P_{1}}}{{\displaystyle P_{0}}}) = 10\log_{10}(\frac{{\displaystyle V_{2 1}}/{\displaystyle Z_{0}}}{{\displaystyle V_{2 0}}/{\displaystyle Z_{1}}})
    \end{equation} 
    
    Se Z1=Z0 então:
    
    \begin{equation}
        {\displaystyle G_{dB}} =10\log_{10}(\frac{{\displaystyle V_{1}}}{{\displaystyle V_{0}}})^{2} = 20\log_{10}|\frac{{\displaystyle V_{1}}}{{\displaystyle V_{0}}}|
    \end{equation}
    
    se Z1 \neq Z0 então 
    
    \begin{equation}
        {\displaystyle G_{dB}} = 20\log_{10}|\frac{{\displaystyle V_{1}}}{{\displaystyle V_{0}}}|
        + 10\log_{10}\frac{{\displaystyle Z_{0}}}{{\displaystyle Z_{1}}}
    \end{equation}
    
\subsection{Medida absoluta de potência em dB
(dBm)}

        O dBm ou dBmW é o nível absoluto de potência em dB, em relação à
    potência de 1mW. É usado em telecomunicações como uma medida de
    potência absoluta devido a sua capacidade de expressar tanto valores
    muito grandes como muito pequenos de uma forma curta. A grande
    vantagem do uso do dBm é que sua medida independe da impedância.
    Para expressar um potência PmW como PdBm usa-se

    \begin{equation}
        {\displaystyle P_{dBm}} = 10\log_{10}\frac{{\displaystyle P_{w}}}{1mW}
    \end{equation}
    
    Quando o valor P x dBm = > 0, então a potência PmW é x dB maior que
1mW. Se P x dBm = < 0, então a potência PmW é x dB menor que 1mW.

Outras medidas de potência absoluta que são raramente usadas:


\begin{center}
    \begin{tabular}{|l|c|}
        \hline  
            dBW & potência absoluta relativa a 1 watt.\\ 
        \hline
            dBf & potência absoluta relativa a 1 femtowatt. \\
        \hline
            dBk & potência absoluta relativa a 1 kilowatt.\\ 
        \hline
        \end{tabular}
    \end{center}

    
\subsection{Cabeçalhos e rodapés customizados}

    Medida absoluta de tensão em dB (dBu)
    
    \begin{equation}
         {\displaystyle G_{dB}} = 20\log_{10}|\frac{{\displaystyle V_{1}}}{{\displaystyle V_{0}}}|
    \end{equation}
    
    substituirmos a tensão V0
 pelo valor 0.775 V que equivale a potência
de 1mW (0dBm) quando aplicado a uma impedância de 600Ω, teremos
uma forma de expressar em valores absolutos a tensão de um ponto do
sistema. A impedância de 600 omega
é o valor padronizado para a maioria
dos circuitos de voz em telefonia pelo ITU-T. A unidade obtida é conhecida por VdBu . A transformação de uma tensão V1
 em dBu é feita
através de:

\begin{equation}
     {\displaystyle G_{dB}} = 20\log_{10}|\frac{{\displaystyle V_{1}}}{1V}|
\end{equation}

As vezes também é usada a abreviação dBv, mas dBu é mais comum
pois dBv é facilmente confundida com dBV que é a medida da tensão
absoluta relativa a 1 volt.


\subsection{Operações com dBm:
}
    Dada uma certa potência absoluta expressa em dBm, a soma (ou
    subtração) de um valor em dB significa, em escala linear, a multiplicação
    (ou divisão) da potência pelo fator correspondente. O resultado é uma
    nova potência absoluta, portanto expressa em dBm.

    \begin{equation}
        {\displaystyle P2_{dBm}} = {\displaystyle P1_{dBm}} + {\displaystyle G_{dB}} \Rightarrow {\displaystyle P2_{w}} = {\displaystyle P1_{w}} \times {\displaystyle G_{\frac{w}{w}}} 
    \end{equation}
    
    Assim, se dobramos uma potência teremos em dB

    \begin{equation}
        {\displaystyle P2_{dBm}} = {\displaystyle P1_{dBm}} + 3dB \Rightarrow {\displaystyle P2_{w}} = {\displaystyle P1_{w}} \times {\displaystyle 2_{\frac{w}{w}}} 
    \end{equation}
    
    Se reduzimos a potência a metade então
    
    \begin{equation}
        {\displaystyle P2_{dBm}} = {\displaystyle P1_{dBm}} - 3dB \Rightarrow {\displaystyle P2_{w}} = \frac{{\displaystyle P1_{w}}}{{\displaystyle 2_{\frac{w}{w}}} } 
    \end{equation}

Ou seja, somar 3dB equivale a dobrar a potência enquanto diminuir 3dB
corresponde reduzir a potência à metade.
A comparação de dois valores expressos em dBm pode ser feita
subtraindo os valores P2,dBm - P1,dBm e obtendo-se a razão entre a potências (P2 / P1 ) em dB. Note que neste caso o resultado é em dB,
pois se trata de uma razão entre potencias e não é uma potência
absoluta

    \begin{equation}
        {\displaystyle G_{dB}} = 10\log_{10}(\frac{{\displaystyle P_{2}}}{\displaystyle P_{1}}) = {\displaystyle P2_{dBm}} - {\displaystyle P1_{dBm}}
    \end{equation}

A subtração de duas potências dadas em dBm resulta no valor em dB
da razão dessas duas potências. O valor de potência em dBm somado
(ou subtraído) à dB resulta num novo valor de potência em dBm. Duas
potências dadas em dBm não podem ser somadas.
Quanto tivermos duas ou mais potências dadas em dBm e quisermos
saber a soma resultante, desde que os sinais que produzem essas
potências sejam descorrelacionados, as potências terão que ser
passadas para a escala linear (w), somas e o resultado retornado para a
escala logarítmica (dBm).

\subsection{Decibel relativo (dBr)}
Esta unidade, denominada dB relativo, é utilizada para indicar a
atenuação ou o ganho em um ponto qualquer de um sistema, em
relação a um ponto de referência do sistema. O ponto de referência é
definido como tendo um nível de 0dBr, e todos os outros pontos tem
seus níveis indicados com níveis relativos a esse de referência. O ponto
de referência pode, em princípio, ser arbitrariamente definido como
sendo qualquer ponto do sistema, ou mesmo fora dele.
Deve-se notar que os níveis relativos não estão relacionados
diretamente com a potência ou amplitudes reais no sistema, podendo
ser indicados mesmo na ausência de qualquer sinal.

\subsection{Potência absoluta do ponto de referência
(dBm0)}
    
    A unidade dBm0 é a potência absoluta, em dBm, medida no ponto de
referência - nível relativo zero do sistema (0dBr). Esta unidade é
normalmente usada para indicar a potência de sinais de níveis fixos tais
como: sinais de teste, tons de sinalização, pilotos, etc. Acrescenta-se o
zero "0" para significar que o nível em dBm corresponde ao valor
medido no ponto de referência.
Em um sistema se o ponto de referência tem um determinado nível
absoluto (por exemplo -20dBm), então se diz que em qualquer ponto do
sistema este sinal tem essa potência em -20dBm0. A potência absoluta
nos diversos pontos do sistema é obtida somando-se a potência dBm0
com a potência dBr do ponto. 



    \begin{equation}
        {\displaystyle P_{A,dBm}} =  {\displaystyle P_{A,dBm0}} +  {\displaystyle P_{A,dBr}}
    \end{equation}
    
Assim por exemplo, um ponto com 5dBr e potência de -20dBm0 terá
-15dBm de potencia absoluta. 

\subsection{Uso de dBu para medir dBm}

Em telecomunicações, para se medir o nível de
potência em dBm de um determinado ponto de um
circuito, normalmente se termina o sistema com
uma carga resistiva igual a impedância nominal do
sistema e mede-se a tensão através de um
voltímetro que tem uma escala calibrada conforme a figura mostrada ao
lado. Se a impedância característica no ponto de teste é de 600omega, a
potência em dBm é a mesma do nível obtido em dBu.

\begin{equation}
        {\displaystyle P_{dBm}} = {\displaystyle V_{dBu}} 
    \end{equation}

Se a impedância for diferente de 600omega, então a potência em dBm será
obtida pela leitura em dBu acrescido do fator de correção K.

\begin{equation}
    {\displaystyle P_{dBm}} = {\displaystyle P_{dBu}} + K
\end{equation}

onde

\begin{equation}
    K = 10\log_{10}\frac{600}{ {\displaystyle Z_{1}}}
\end{equation}

A tabela abaixo mostra o fator de correção K para alguns valores de
impedância

\input{tabela.tex}

\section*{Considerações finais}
\addcontentsline{toc}{section}{Considerações finais}



% ----------------------------------------------------------
% ELEMENTOS PÓS-TEXTUAIS
% ----------------------------------------------------------
\postextual

% ---
% Título e resumo em língua estrangeira
% ---

% \twocolumn[    		% INICIO DE ARTIGO EM DUAS COLUNAS

% titulo em inglês
\titulo{Canonical academic article model with \abnTeX}
\emptythanks
\maketitle

% resumo em português
\renewcommand{\resumoname}{Abstract}
\begin{resumoumacoluna}
 \begin{otherlanguage*}{english}
   According to ABNT NBR 6022:2003, an abstract in foreign language is a back
   matter mandatory element.

   \vspace{\onelineskip}
 
   \noindent
   \textbf{Key-words}: latex. abntex.
 \end{otherlanguage*}  
\end{resumoumacoluna}

% ]  				% FIM DE ARTIGO EM DUAS COLUNAS
% ---

% ----------------------------------------------------------
% Referências bibliográficas
% ----------------------------------------------------------


% ----------------------------------------------------------
% Glossário
% ----------------------------------------------------------
%
% Há diversas soluções prontas para glossário em LaTeX. 
% Consulte o manual do abnTeX2 para obter sugestões.
%
%\glossary

% ----------------------------------------------------------
% Apêndices
% ----------------------------------------------------------

% ---
% Inicia os apêndices
% ---

\end{document}
